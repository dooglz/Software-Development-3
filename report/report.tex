%%% This LaTeX source document can be used as the basis for your technical
%%% report. Intentionally stripped and simplified
%%% and commands should be adjusted for your particular paper - title, 
%%% author, citations, equations, etc.
% % Citations/references are in report.bib 

\documentclass[conference]{acmsiggraph}

\usepackage{graphicx}
\graphicspath{{./images/}}
\newcommand{\figuremacroW}[4]{
	\begin{figure}[h] %[htbp]
		\centering
		\includegraphics[width=#4\columnwidth]{#1}
		\caption[#2]{\textbf{#2} - #3}
		\label{fig:#1}
	\end{figure}
}

\newcommand{\figuremacroF}[4]{
	\begin{figure*}[h] % [htbp]
		\centering
		\includegraphics[width=#4\textwidth]{#1}
		\caption[#2]{\textbf{#2} - #3}
		\label{fig:#1}
	\end{figure*}
}
\usepackage{xcolor}
\definecolor{lbcolor}{rgb}{0.98,0.98,0.98}
\usepackage{listings}

\lstset{
	escapeinside={/*@}{@*/},
	language=C++,
	%basicstyle=\small\sffamily,
	%basicstyle=\small\sffamily,	
	basicstyle=\fontsize{8.5}{12}\selectfont,
	%basicstyle=\small\ttfamily,
	%basicstyle=\scriptsize, % \footnotesize,
	%basicstyle=\footnotesize,
	%keywordstyle=\color{blue}\bfseries,
	%basicstyle= \listingsfont,
	numbers=left,
	numbersep=2pt,    
	xleftmargin=2pt,
	%numberstyle=\tiny,
	frame=tb,
	%frame=single,
	columns=fullflexible,
	showstringspaces=false,
	tabsize=4,
	keepspaces=true,
	showtabs=false,
	showspaces=false,
	%showstringspaces=true
	backgroundcolor=\color{lbcolor},
	morekeywords={inline,public,class,private,protected,struct},
	captionpos=t,
	lineskip=-0.4em,
	aboveskip=10pt,
	%belowskip=50pt,
	extendedchars=true,
	breaklines=true,
	prebreak = \raisebox{0ex}[0ex][0ex]{\ensuremath{\hookleftarrow}},
	keywordstyle=\color[rgb]{0,0,1},
	commentstyle=\color[rgb]{0.133,0.545,0.133},
	stringstyle=\color[rgb]{0.627,0.126,0.941}
}

\usepackage{lipsum}

\TOGonlineid{45678}
\TOGvolume{0}
\TOGnumber{0}
\TOGarticleDOI{1111111.2222222}
\TOGprojectURL{}
\TOGvideoURL{}
\TOGdataURL{}
\TOGcodeURL{}

\title{Coursework Report}

\author{Sam Serrels\\\ 40082367@napier.ac.uk \\
Edinburgh Napier University\\
Software Development 3 (SET09101)}
\pdfauthor{Sam Serrels}

\keywords{Multi-platform Physics, Optimisation, GPU, Cell, PS3}

\begin{document}

\maketitle

\section{Introduction}

Realistic Real time physics simulation is highly sought after in interactive applications, especially games. Achieving high-accuracy while maintaining performance in often resource restricted environments(I.E a games console) requires the highest level of optimisations and often results in a trade-off with simulation speed against Accuracy. This project attempts to record and analyse the performance of various optimisations on a simulated scene. This will be taken further by applying the project to various different processing architectures. The scene that will be simulated is a large set of Bouncy balls, travelling down a hill.


\section{Design Patterns}
\paragraph{Factory}
Trying to regain performance from an external physics engine can be a hard task, diving into the source code requires expert knowledge of the inner-workings of the whole system. A common path is to shape the design of the game code to conform better to the demands of the physics engine and hope that the internal optimisations  will be sufficient. Often enough, they are not.

\paragraph{Factory}
Trying to regain performance from an external physics engine can be a hard task, diving into the source code requires expert knowledge of the inner-workings of the whole system. A common path is to shape the design of the game code to conform better to the demands of the physics engine and hope that the internal optimisations  will be sufficient. Often enough, they are not.

\paragraph{Command}
Trying to regain performance from an external physics engine can be a hard task, diving into the source code requires expert knowledge of the inner-workings of the whole system. A common path is to shape the design of the game code to conform better to the demands of the physics engine and hope that the internal optimisations  will be sufficient. Often enough, they are not.


\section{Threads}
\paragraph{Threads}
Trying to regain performance from an external physics engine can be a hard task, diving into the source code requires expert knowledge of the inner-workings of the whole system. A common path is to shape the design of the game code to conform better to the demands of the physics engine and hope that the internal optimisations  will be sufficient. Often enough, they are not.


\section{Networking and Webgui}

\paragraph{Optimising for Physics Engines}
Trying to regain performance from an external physics engine can be a hard task, diving into the source code requires expert knowledge of the inner-workings of the whole system. A common path is to shape the design of the game code to conform better to the demands of the physics engine and hope that the internal optimisations  will be sufficient. Often enough, they are not.

\figuremacroW
{screen2}
{Bullet Physics PS3 Pipeline}
{Requires Intermediate Data Swapping Between PPU and SPU}
{1.0}



\figuremacroF
{screen1}
{Sequential Clark Wright implementation results}
{Requires Intermediate Data Swapping Between PPU and SPU}
{1.0}

\figuremacroF
{uml1}
{Sequential Clark Wright implementation results}
{Requires Intermediate Data Swapping Between PPU and SPU}
{1.0}

\section{Conclusions}

\paragraph{Optimising for Physics Engines}
Trying to regain performance from an external physics engine can be a hard task, diving into the source code requires expert knowledge of the inner-workings of the whole system. A common path is to shape the design of the game code to conform better to the demands of the physics engine and hope that the internal optimisations  will be sufficient. Often enough, they are not.

\clearpage
\end{document}

